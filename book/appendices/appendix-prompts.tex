\chapter{Prompt Templates}
\label{appendix:prompts}

This appendix provides ready-to-use prompt templates for common business tasks. Copy and customize these for your specific needs.

\section{Customer Service}

\subsection{Complaint Response}

\begin{promptexample}[Template: Complaint Response]
Customer complaint:
[paste complaint]

Context:
- Our policy on this issue: [policy]
- What we can offer: [available resolutions]
- Tone: empathetic, professional, solution-focused

Draft a response that:
1. Acknowledges their frustration without being defensive
2. Takes appropriate responsibility
3. Offers a clear resolution or next steps
4. Invites further dialogue if needed
\end{promptexample}

\subsection{Information Request}

\begin{promptexample}[Template: Information Request Response]
Customer question:
[paste question]

Relevant information:
- [fact 1]
- [fact 2]
- [fact 3]

Draft a response that:
1. Answers the question directly
2. Provides helpful context
3. Anticipates follow-up questions
4. Offers additional assistance
\end{promptexample}

\section{Meeting and Documentation}

\subsection{Meeting Summary}

\begin{promptexample}[Template: Meeting Summary]
Meeting transcript/notes:
[paste transcript or notes]

Create a meeting summary with:

1. Overview (2-3 sentences: what was this meeting about?)

2. Decisions Made
- [Decision with rationale if discussed]

3. Action Items
| Action | Owner | Due Date |
| --- | --- | --- |

4. Key Discussion Points (3-5 bullets)

5. Open Questions / Parking Lot

6. Next Steps

Only include information actually discussed. Do not invent details.
\end{promptexample}

\subsection{Document Summary}

\begin{promptexample}[Template: Document Summary]
Document:
[paste document or describe]

Summarize for: [audience - executive, team member, external party]
Purpose: [why they need this summary]

Provide:
1. Executive summary (50 words max)
2. Key points (5-7 bullets)
3. Important caveats or limitations
4. Recommended next steps (if applicable)
\end{promptexample}

\section{Analysis and Research}

\subsection{Research Brief}

\begin{promptexample}[Template: Research Brief]
I need to understand [topic] for [purpose].
I have [time available] to review this.
My background: [relevant context about what you already know]

Provide:
1. Overview (2-3 paragraphs for someone new to this)
2. Key concepts I must understand
3. Common misconceptions to avoid
4. Most important recent developments
5. Questions I should be asking

Cite specific sources where possible.
\end{promptexample}

\subsection{Decision Analysis}

\begin{promptexample}[Template: Decision Analysis]
Decision: [what needs to be decided]

Options:
- Option A: [description]
- Option B: [description]
- Option C: [description]

Stakeholders: [who cares]
Constraints: [budget, time, resources, politics]
Goals: [what success looks like]

For each option, analyze:
1. Pros
2. Cons
3. Risks
4. Resources required
5. Reversibility

Recommend: [which option and why]
\end{promptexample}

\subsection{Pre-Mortem Analysis}

\begin{promptexample}[Template: Pre-Mortem]
I am planning to [decision/action].

Imagine it is 12 months from now and this has failed badly.
- What went wrong?
- What early warning signs did we miss?
- What assumptions turned out to be false?
- What did competitors do that we did not anticipate?

Now: What can we do today to prevent these failures?
\end{promptexample}

\section{Content Creation}

\subsection{Content Variation}

\begin{promptexample}[Template: Content Variation]
Source content:
[paste original]

Create 5 variations optimized for:
1. LinkedIn (professional, insight-led)
2. Twitter/X (concise, engaging hook)
3. Email subject line (curiosity-driven, under 50 chars)
4. Email body (scannable, clear CTA)
5. Blog intro paragraph (SEO-aware, hooks reader)

Maintain core message but adapt format and tone for each channel.
\end{promptexample}

\subsection{Persona Adaptation}

\begin{promptexample}[Template: Persona Adaptation]
Original message:
[paste content]

Adapt for these audiences (same core message, different framing):
1. CFO (focus: ROI, risk, bottom line)
2. IT Director (focus: implementation, integration, security)
3. End User (focus: ease of use, daily benefits)

Each version: 100-150 words.
\end{promptexample}

\section{Project Management}

\subsection{Project Kickoff Structure}

\begin{promptexample}[Template: Project Kickoff]
New project: [name/description]
Sponsor: [who]
Rough timeline expectation: [duration]
Budget range: [if known]
Key stakeholders: [list]

Help me structure the kickoff:
1. What questions must be answered before we start?
2. What scope clarifications are needed?
3. What are likely risks we should discuss early?
4. What dependencies might not be obvious?
5. Suggested agenda for kickoff meeting (60 minutes)
\end{promptexample}

\subsection{Status Update}

\begin{promptexample}[Template: Status Update]
Project: [name]
Audience: [who will read this]
Update frequency: [weekly/monthly]

Raw inputs:
[paste task status, blockers, metrics, etc.]

Create status update:
- Executive summary (2-3 sentences)
- Progress against milestones
- Key accomplishments this period
- Blockers and risks
- Upcoming focus areas
- Asks (if any)
\end{promptexample}

\section{Data Analysis}

\subsection{Data Interpretation}

\begin{promptexample}[Template: Data Interpretation]
Data:
[paste or describe dataset]

Context: [what this data represents]
Question: [what you want to understand]

Provide:
1. Summary statistics (if numerical)
2. Key patterns or trends
3. Outliers or anomalies
4. Possible explanations
5. Recommended additional analysis
6. Caveats about this interpretation
\end{promptexample}

\subsection{Feedback Analysis}

\begin{promptexample}[Template: Feedback Analysis]
Analyze the following customer feedback items.

For each item, provide:
1. Sentiment (Positive/Neutral/Negative)
2. Category (from: [your categories])
3. Priority (High/Medium/Low)
4. Topic (brief label, max 3 words)
5. Key phrase (most important quote)

Format output as a table suitable for spreadsheet import.

Feedback items:
[paste 20-30 items]
\end{promptexample}

\section{Tips for Using These Templates}

\begin{enumerate}
    \item \textbf{Customize the context.} These templates are starting points. Add details specific to your situation.

    \item \textbf{Specify the format.} If you need output in a specific format (table, bullets, paragraphs), say so explicitly.

    \item \textbf{Include constraints.} Length limits, tone requirements, and must-avoid topics all improve results.

    \item \textbf{Iterate.} If the first output is not right, refine your prompt rather than starting over.

    \item \textbf{Verify always.} No template guarantees accuracy. Check all outputs before using them.
\end{enumerate}
