\chapter{Preface}

Every week, someone asks me whether their organization should ``do something with AI.'' The question comes from CEOs worried about competitive positioning, division heads curious about productivity gains, and board members wondering if they are missing a strategic opportunity. The honest answer is nuanced: yes, you probably should, but not the way most organizations are approaching it.

I wrote this book because I got tired of two things: the breathless hype that treats AI as magic, and the dismissive skepticism that treats it as a passing fad. Both miss the point. AI tools are powerful capabilities with real limitations. Learning to deploy them effectively is a leadership skill, and like any skill, it can be developed.

This is not a technical book. You do not need to understand neural networks, machine learning architectures, or data science. What you need to understand is what AI can and cannot do for your business, how to deploy it without creating organizational risk, and how to evaluate whether it is actually delivering value.

The examples in this book come from real business situations across industries. I have watched marketing teams cut content production time in half. I have seen executives present AI-generated analysis to boards, only to discover it contained fabricated statistics. I have observed customer service organizations delight customers with AI assistance and frustrate them with poorly implemented automation. The difference between success and failure is rarely the technology---it is how leaders deploy it.

What you will find here:

\begin{itemize}
    \item A clear mental model for what AI actually does, explained in business terms
    \item Practical techniques for using AI in everyday business operations
    \item Frameworks for evaluating AI investments and measuring ROI
    \item Real examples of what works, what fails, and why
    \item Templates and approaches you can adapt to your organization
\end{itemize}

The goal is not to use AI everywhere---it is to use it where it creates value and avoid it where it creates risk. Throughout this book you will learn specific tools and frameworks, but the underlying patterns matter more than any current tool.

By the end of this book, you should be able to look at any situation and quickly assess: Can AI help here? What is the business case? What governance do I need?

Your competitors are learning how to deploy AI strategically. Let's make sure you are ahead of them.
