\chapter{Preface}

Every week, someone asks me whether they should ``do something with AI.'' The question comes from executives worried about falling behind, managers curious about productivity gains, and team leads wondering if they are missing something important. The honest answer is complicated: yes, you probably should, but not the way most people are approaching it.

I wrote this book because I got tired of two things: the breathless hype that treats AI as magic, and the dismissive skepticism that treats it as useless. Both miss the point. AI assistants are powerful tools with real limitations. Learning to work with them effectively is a skill, and like any skill, it can be taught.

This is not a technical book. You do not need to understand neural networks, transformers, or machine learning algorithms. What you need to understand is what AI can and cannot do, how to use it without embarrassing yourself or your organization, and how to evaluate whether it is actually helping.

The examples in this book come from real business situations. I have watched marketing teams cut content production time in half. I have seen executives get trapped by AI-generated reports full of confident-sounding nonsense. I have observed customer service teams delight customers with AI assistance and frustrate them with poorly implemented chatbots. The difference between success and failure is rarely the technology---it is how people use it.

What you will find here:

\begin{itemize}
    \item A clear mental model for what AI actually does, without the jargon
    \item Practical techniques for using AI in everyday work
    \item Frameworks for evaluating AI projects and measuring results
    \item Real examples of what works and what fails
    \item Templates and prompts you can adapt to your own needs
\end{itemize}

What you will not find:

\begin{itemize}
    \item Technical deep dives into how AI systems work internally
    \item Promises about AI revolutionizing everything
    \item Advice requiring data science expertise
    \item Speculation about artificial general intelligence
\end{itemize}

The competitive landscape is shifting. Organizations that learn to use AI effectively are pulling ahead of those that do not. But ``effectively'' is the key word. The goal is not to use AI everywhere---it is to use it where it helps and avoid it where it does not.

By the end of this book, you should be able to look at any business task and quickly assess: Can AI help here? How should I approach it? What verification do I need? And when is it faster to just do it the old way?

Let's get started.
