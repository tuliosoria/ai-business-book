\chapter{AI in Action: Upgrading Your Workflow}

Here's the truth about GenAI in product work: it doesn't replace thinking; it compresses the time between thinking and trying.

You can draft a PRD faster, generate ten variants of microcopy, prototype flows, summarize research, and turn support tickets into themes. But if your strategy is unclear, AI just helps you ship confusion faster.

This chapter is about integrating AI into your actual workflow—the daily work of specs, research, prioritization, and communication. I'll show you where it works, where it fails, and how to use it without getting sloppy.

\section{The Workflow Integration Principle}

Before we dive into specifics, let me share a principle that guides my approach:

AI accelerates throughput. It doesn't fix fundamentals.

If you know what you're building and why, AI makes you faster. If you don't, AI makes you faster at going nowhere. The leverage is real, but it's leverage on your existing direction. Make sure your direction is right before you accelerate.

With that framing, let's look at specific applications.

\section{Drafting Documents}

This is where most PMs start, and for good reason. Drafting is time-consuming, and AI dramatically reduces the time to first draft.

\subsection{PRDs and Specs}

When I'm writing a PRD now, I start with one sentence and refuse to move on until it's sharp: What user, what moment, what pain, what outcome.

Once I have that sentence, I can use AI to expand it into structure. Here's a typical workflow:

\textbf{Step 1: Write the core sentence myself.} No AI here. This is strategy work, and I need to own it.

\textbf{Step 2: Generate a PRD skeleton.} I give AI my core sentence and ask for an outline. What sections should this PRD include? What questions should each section answer?

\textbf{Step 3: Draft each section with AI assistance.} For each section, I provide context and ask AI to generate a first pass. I'm specific about format, length, and what to include.

\textbf{Step 4: Review and revise heavily.} The AI draft is a starting point, not an endpoint. I rewrite sections that don't match my intent. I add context AI couldn't know. I verify any factual claims.

\textbf{Step 5: Share and iterate.} The document goes to stakeholders. I use AI to help process their feedback and generate revisions.

The time savings are real—maybe 50-60\% reduction in drafting time. But the quality depends entirely on the clarity of my initial thinking. If my one sentence is vague, the PRD is vague. AI amplifies whatever you feed it.

\subsection{User Stories and Acceptance Criteria}

User stories are a great use case for AI because they follow predictable patterns. Given a feature description, generating user stories is largely mechanical.

My approach:
\begin{itemize}
    \item Describe the feature and its goals
    \item Ask AI to generate user stories in a specific format
    \item Ask AI to generate acceptance criteria for each story
    \item Review for completeness and accuracy
\end{itemize}

The review step is critical. AI-generated acceptance criteria often miss edge cases. They assume happy paths. They don't know your system's constraints. Use AI to generate the first pass, then apply your judgment to catch what's missing.

\subsection{Communication Documents}

Status updates, stakeholder emails, meeting summaries—these are high-volume, low-complexity documents. Perfect for AI acceleration.

I'll often dump raw notes into AI and ask for a structured summary. Or I'll describe the situation and ask for a draft email. The output usually needs editing for tone and specificity, but it's much faster than starting from blank.

\section{Research and Analysis}

AI changes how you process information. You can handle more inputs, synthesize faster, and identify patterns that would take hours to find manually.

\subsection{Interview Synthesis}

Traditional approach: conduct interviews, transcribe notes, read through everything, identify themes, write a synthesis. This could take days for a substantial research project.

AI-assisted approach: feed transcripts to AI, ask for themes and patterns, generate a first-pass synthesis, then validate against the raw material.

The time savings are dramatic—hours instead of days. But there's a risk: AI synthesis can miss nuance. It picks up what's explicit and frequent. It can miss what's implied, unusual, or emotionally significant.

My rule: AI can summarize customer interviews, but I still check the raw notes when the decision is expensive. For exploratory research, AI synthesis is a great starting point. For decisions with high stakes, I verify against primary sources.

\subsection{Competitive Analysis}

AI can help you structure competitive analysis and fill in publicly available information. It's particularly good at:
\begin{itemize}
    \item Generating comparison frameworks (what dimensions should we compare?)
    \item Summarizing competitor positioning from their public materials
    \item Identifying gaps or patterns across competitors
\end{itemize}

The limitation: AI doesn't know what competitors announced yesterday. It doesn't have access to pricing that requires a sales call. It doesn't know insider information. Use it for structure and publicly available synthesis, then fill in the gaps with primary research.

\subsection{Market Research}

Similar dynamics. AI can help you:
\begin{itemize}
    \item Structure your research questions
    \item Summarize industry reports and articles
    \item Generate hypotheses to test
    \item Identify related trends or analogous markets
\end{itemize}

Again, verification matters. AI-generated market statistics should be checked against primary sources. The model doesn't know which numbers are accurate and which are hallucinated.

\section{Ideation and Brainstorming}

This is one of AI's strongest applications: generating options you wouldn't have thought of.

\subsection{Feature Brainstorming}

When you're stuck, AI can help you explore the solution space. Give it the problem you're trying to solve and ask for ten different approaches. Or twenty. Or fifty.

Most of the ideas will be mediocre or irrelevant. That's fine. You're not looking for AI to solve the problem—you're looking for AI to expand your thinking. One unexpected idea in twenty is worth the two minutes it takes to generate them.

\subsection{Edge Case Generation}

The most common failure mode in product isn't a bad idea. It's a good idea with undefined edges. AI is surprisingly good at finding edge cases.

Give AI a feature description and ask: ``What could go wrong? What edge cases should we consider? What assumptions are we making that might not hold?''

The output won't be comprehensive, but it will often surface cases you hadn't considered. Use it as a checklist to review, not a replacement for your own analysis.

\subsection{Variant Generation}

Need multiple versions of something? AI excels here.
\begin{itemize}
    \item Ten variations of a tagline
    \item Five ways to phrase an error message
    \item Three different onboarding flows
    \item Multiple framings of a pricing page
\end{itemize}

Generate variants quickly, then apply judgment to select and refine. This is where the cheap artifact principle pays off: when prototypes are cheap, you can explore more options before committing.

\section{Clustering and Prioritization}

When you have a lot of inputs—feedback, feature requests, support tickets—AI can help you organize them.

\subsection{Feedback Clustering}

Dump a batch of customer feedback into AI and ask it to cluster by theme. You'll get a first-pass organization that you can refine.

This works well for:
\begin{itemize}
    \item NPS comments
    \item Support ticket themes
    \item Feature requests
    \item App store reviews
    \item Survey responses
\end{itemize}

The clustering won't be perfect. You'll want to review, merge some categories, split others. But it's much faster than reading through everything and clustering manually.

\subsection{Prioritization Support}

AI can't prioritize for you—prioritization requires judgment about strategy, constraints, and trade-offs that AI doesn't have. But it can support your prioritization process:

\begin{itemize}
    \item \textbf{Scoring frameworks}: Ask AI to apply a scoring rubric to a list of features, based on criteria you define
    \item \textbf{Trade-off analysis}: Describe two options and ask AI to articulate the trade-offs
    \item \textbf{Impact estimation}: Given information about a feature and your user base, ask AI to estimate potential impact
\end{itemize}

Use these as inputs to your decision, not as the decision itself. AI can help you think more rigorously, but the final call is yours.

\section{Communication Efficiency}

A lot of PM work is communication. AI can make it more efficient.

\subsection{Meeting Preparation}

Before important meetings, I'll often use AI to:
\begin{itemize}
    \item Generate an agenda based on topics to cover
    \item Draft talking points for difficult conversations
    \item Anticipate questions and prepare responses
    \item Summarize relevant context that attendees need
\end{itemize}

This isn't about scripting conversations—it's about walking in prepared rather than improvising.

\subsection{Asynchronous Communication}

Written communication is increasingly important in distributed teams. AI helps you:
\begin{itemize}
    \item Draft clearer, more structured messages
    \item Adjust tone for different audiences
    \item Summarize long threads for people joining late
    \item Generate FAQ responses for common questions
\end{itemize}

When I'm writing, I try to avoid the temptation to sound smart. I'm not writing to impress; I'm writing to be understood. AI can help with this—ask it to simplify your draft, remove jargon, or make it more direct.

\subsection{Documentation Maintenance}

Documentation rots. Information gets outdated. Nobody has time to update it. AI can help:
\begin{itemize}
    \item Identify inconsistencies between documents
    \item Update boilerplate sections across multiple docs
    \item Generate FAQs from existing documentation
    \item Convert documentation between formats
\end{itemize}

It's not glamorous work, but it's work that often doesn't get done. If AI makes it faster, it might actually happen.

\section{Where AI Gets You in Trouble}

Let me be direct about the failure modes.

\subsection{Generating Without Strategy}

If you can generate ten flows in a day, you can also generate ten wrong flows in a day. The constraint shifts from production capacity to judgment.

Don't use AI to generate more. Use AI to generate faster so you have more time to think. If you find yourself producing artifacts without a clear reason, stop. Go back to basics. What user? What moment? What pain? What outcome?

\subsection{Skipping Verification}

AI outputs need to be checked. Every time. The more you rely on AI without verification, the more errors slip through.

Build verification into your workflow. First pass: generate. Second pass: verify. Make this explicit and non-negotiable.

\subsection{Over-Polishing First Drafts}

AI can generate polished-looking outputs quickly. That's dangerous because polished outputs feel finished. They resist revision.

I'd rather have a rough draft that's clearly a draft than a polished draft that gets accepted without critical review. Sometimes I deliberately ask AI for an outline or bullet points rather than prose, to make clear that work remains.

\subsection{Outsourcing Thinking}

The moment you treat AI output as truth, you get sloppy. The moment you treat it as a junior assistant that works fast but hallucinates sometimes, you get value.

AI is not a replacement for your thinking. It's a tool that accelerates certain parts of your work while requiring judgment on others. If you find yourself accepting AI outputs without engaging critically, you're doing it wrong.

\section{Building Your Personal System}

Here's how I think about AI in my workflow:

\subsection{High-AI Activities}
\begin{itemize}
    \item First drafts of documents
    \item Summarization and synthesis
    \item Variant generation
    \item Feedback clustering
    \item Communication polishing
\end{itemize}

\subsection{Low-AI Activities}
\begin{itemize}
    \item Strategy definition
    \item Priority decisions
    \item Stakeholder negotiations
    \item Final sign-off on anything customer-facing
    \item Judgment calls with incomplete information
\end{itemize}

\subsection{Always-Verify Activities}
\begin{itemize}
    \item Any factual claims
    \item Numbers and statistics
    \item Technical specifications
    \item Quotes or citations
    \item Customer-facing content
\end{itemize}

This isn't a rigid framework. It's a starting point. Build your own system based on what works for your context, your risk tolerance, and your strengths.

\section{The Bottom Line}

AI in action is about integration, not replacement. You're adding a capability to your existing workflow, not substituting AI for your judgment.

The gains are real:
\begin{itemize}
    \item Faster drafting
    \item Broader brainstorming
    \item More efficient research
    \item Quicker communication
\end{itemize}

The risks are also real:
\begin{itemize}
    \item Unverified errors
    \item Strategy drift
    \item Over-production of artifacts
    \item Skill atrophy
\end{itemize}

Navigate between them by staying anchored in fundamentals. Know your strategy. Verify your outputs. Maintain your skills. Use AI for throughput, not for thinking.

When I'm deciding what to cut, I ask one question: what gives us learning? Not what looks impressive. Not what wins the demo. What gives us signal.

That principle applies to AI too. Use it for what gives you learning. Skip it when it just adds noise.
