\chapter{Five Simple AI Projects You Can Start Today}

\epigraph{The journey of a thousand miles begins with a single step.}{Lao Tzu}

This chapter provides five concrete, low-risk projects that non-technical teams can implement quickly using existing AI tools and basic computer skills. Each project includes step-by-step implementation guides, templates, and measurement approaches.

\section{Project 1: Automated Email and Message Response Templates}

\textbf{Business Value:} 5-10 hours per week saved for high-volume communicators

\textbf{Skills Needed:} Beginner (email, basic document tools)

\textbf{Tools Required:} ChatGPT, Claude, or similar; document storage

\textbf{Timeline:} 1-2 weeks

\subsection{Week 1: Identify and Create Templates}

\textbf{Day 1-2: Identify Your Templates}
\begin{enumerate}
    \item Review sent emails from the past month
    \item List the 10 most frequent message types
    \item Prioritize: start with the 5 most common
\end{enumerate}

Common message types include:
\begin{itemize}
    \item Scheduling meeting requests
    \item Following up on unanswered emails
    \item Providing standard information (pricing, specs, policies)
    \item Acknowledging receipt with timeline
    \item Declining requests politely
\end{itemize}

\textbf{Day 3-4: Create Reference Examples}

For each message type:
\begin{enumerate}
    \item Find a good example you have written before
    \item Note what varies (names, dates, specifics)
    \item Note what stays consistent (structure, tone)
\end{enumerate}

\textbf{Day 5: Generate Templates}

\begin{promptexample}[Template Generation Prompt]
I need an email template for: [type, e.g., ``following up on a proposal sent 1 week ago'']

Context:
- My role: [your role]
- Typical recipient: [their role]
- Our company tone: [e.g., ``professional but friendly, not stiff'']
- Typical scenario: [describe situation]

Example of how I have written this before:
[paste example]

Create a template with:
- Clear placeholder markers [LIKE THIS] for variable content
- Multiple options for opening lines (I can pick based on relationship)
- A version for first follow-up and second follow-up
- Appropriate length for [email/Slack/Teams]
\end{promptexample}

\subsection{Week 2: Publish and Measure}

\begin{enumerate}
    \item Create quick reference guide (one-pager)
    \item Train team in 15-minute session
    \item Track usage (simple survey)
    \item Measure response times before/after
\end{enumerate}

\begin{table}[htbp]
\centering
\begin{tabular}{lll}
\toprule
\textbf{Metric} & \textbf{How to Measure} & \textbf{Target} \\
\midrule
Time to respond & Track 20 messages before/after & -50\% \\
Template adoption & Weekly survey: ``Used templates?'' & $>$80\% \\
Response quality & Recipient feedback or reply rates & No decrease \\
\bottomrule
\end{tabular}
\caption{Metrics for email template project}
\end{table}

\begin{tip}[Avoiding Robotic Templates]
Add this instruction to your prompts: ``Vary sentence structure, avoid obvious template language, include one specific detail placeholder that forces personalization.''
\end{tip}

\section{Project 2: Weekly Report and Status Update Automation}

\textbf{Business Value:} 2-5 hours per week saved writing reports

\textbf{Skills Needed:} Beginner to Intermediate

\textbf{Tools Required:} ChatGPT or Claude; spreadsheet; document tools

\textbf{Timeline:} 2-3 weeks

\subsection{Week 1: Design Your Report Structure}

\textbf{Day 1-2: Analyze Current Reports}
\begin{enumerate}
    \item Collect 4 weeks of your current status reports
    \item Identify consistent sections and variable content
    \item Note what stakeholders actually read and care about
    \item Identify data sources for each section
\end{enumerate}

\textbf{Day 3-4: Create Report Template}

Standard sections for a weekly status report:
\begin{enumerate}
    \item Executive Summary (2-3 sentences)
    \item Key Accomplishments (bulleted)
    \item Progress Against Goals (metrics or milestones)
    \item Blockers and Risks (if any)
    \item Priorities for Next Week
    \item Asks/Decisions Needed (if any)
\end{enumerate}

\textbf{Day 5: Build Input Collection Process}

Create a simple input form or template:
\begin{itemize}
    \item What did you complete this week?
    \item What metrics moved?
    \item What is blocking you?
    \item What is planned for next week?
\end{itemize}

\subsection{Week 2: Automate the Assembly}

\begin{promptexample}[Report Assembly Prompt]
You are helping create a weekly status report.

Report template:
[paste your template]

This week's raw inputs:
[paste collected inputs]

Metrics:
[paste any numbers/data]

Create a draft report that:
- Summarizes accomplishments in crisp, active language
- Highlights only the most important 3-5 items
- Uses specific numbers where available
- Notes blockers clearly with proposed solutions or asks
- Keeps executive summary under 50 words
- Maintains professional tone appropriate for [VP/Director/C-suite audience]

DO NOT invent accomplishments. Only summarize what is in the inputs. If inputs are unclear, flag them rather than making assumptions.
\end{promptexample}

\begin{warning}[Critical Verification]
Always verify that the AI has not invented accomplishments or metrics. This is the most common failure mode for report automation.
\end{warning}

\section{Project 3: Customer Feedback Analysis Dashboard}

\textbf{Business Value:} Understand customer pain points without manual review; save 10-20 hours per month

\textbf{Skills Needed:} Intermediate

\textbf{Tools Required:} ChatGPT or Claude; spreadsheet; optional visualization tool

\textbf{Timeline:} 3-4 weeks

\subsection{Week 1: Data Collection and Preparation}

\textbf{Day 1-2: Identify Feedback Sources}
\begin{itemize}
    \item Customer support tickets
    \item Survey responses (NPS, CSAT)
    \item Product reviews
    \item Social media mentions
    \item Sales call notes
\end{itemize}

\textbf{Day 3-4: Export and Clean Data}

For each source:
\begin{enumerate}
    \item Export to CSV or copy to spreadsheet
    \item Remove personal information (names, emails, phone numbers)
    \item Add source identifier column
    \item Standardize date format
    \item Target: 100-500 feedback items for initial analysis
\end{enumerate}

\textbf{Day 5: Create Analysis Categories}

Define your taxonomy:
\begin{itemize}
    \item \textbf{Sentiment:} Positive, Neutral, Negative
    \item \textbf{Category:} Product Quality, Customer Service, Pricing, Delivery, Website/App, Other
    \item \textbf{Priority:} High (churn risk), Medium (dissatisfied), Low (suggestion)
    \item \textbf{Topic:} 10-15 topics specific to your business
\end{itemize}

\subsection{Week 2: Build the Analysis Process}

\begin{promptexample}[Feedback Analysis Prompt]
Analyze the following customer feedback items.

For each item, provide:
1. Sentiment (Positive/Neutral/Negative)
2. Category (from: Product Quality, Customer Service, Pricing, Delivery, Website/App, Other)
3. Priority (High=churn risk, Medium=dissatisfied, Low=suggestion)
4. Topic (brief label, max 3 words)
5. Key phrase (the most important quote from the feedback)

Format output as a table suitable for pasting into a spreadsheet.

Feedback items:
[paste 20-30 items at a time]
\end{promptexample}

\textbf{Process in batches:}
\begin{enumerate}
    \item Run 20-30 items at a time through the prompt
    \item Paste results into master spreadsheet
    \item Spot-check 10\% of items for accuracy
    \item Refine prompt if accuracy is below 85\%
\end{enumerate}

\subsection{Week 3: Create Summary and Dashboard}

\begin{promptexample}[Summary Generation Prompt]
Based on the following categorized feedback data:
[paste summary or full dataset]

Provide:
1. Sentiment breakdown (\% positive, neutral, negative)
2. Top 5 categories by volume
3. Top 5 topics mentioned
4. High-priority issues requiring immediate attention
5. Notable quotes that represent common themes
6. Recommended actions (3-5 specific suggestions)
\end{promptexample}

\section{Project 4: Knowledge Base with AI-Generated Summaries}

\textbf{Business Value:} Faster onboarding, reduced repetitive questions, searchable knowledge

\textbf{Skills Needed:} Intermediate

\textbf{Tools Required:} Notion, wiki, or document platform; ChatGPT or Claude

\textbf{Timeline:} 3-4 weeks

\subsection{Week 1: Audit and Structure}

\textbf{Day 1-2: Content Inventory}

Create a spreadsheet with columns:
\begin{itemize}
    \item Title
    \item Type (process, how-to, policy, reference, training)
    \item Location
    \item Last Updated
    \item Owner
    \item Summary Exists?
\end{itemize}

\textbf{Day 3-4: Define Structure}

Design your library hierarchy:
\begin{verbatim}
Company Knowledge Base
|-- Getting Started (Onboarding)
|-- How We Work (Processes)
|-- Product Information
|-- Policies and Guidelines
|-- Tools and Systems
|-- Team Resources
+-- External Resources
\end{verbatim}

\textbf{Day 5: Prioritize}
\begin{enumerate}
    \item Rank content by frequency of use/questions
    \item Identify top 20 items to summarize first
    \item Schedule remaining content for later batches
\end{enumerate}

\subsection{Week 2: Generate Summaries}

\begin{promptexample}[Knowledge Base Entry Prompt]
Create a knowledge base entry for the following document.

Document title: [title]
Document type: [type]
Audience: [who needs this]

Content:
[paste document content]

Generate:
1. One-sentence summary (what is this document?)
2. Key points (3-5 bullets of most important information)
3. When to use this (what situations call for this document?)
4. Related topics (what else might someone looking at this need?)
5. Keywords for search (10-15 terms people might search)

Keep language clear and simple. Avoid jargon unless defining it.
\end{promptexample}

\subsection{Week 3-4: Build and Launch}

\begin{enumerate}
    \item Set up platform (Notion, wiki, etc.)
    \item Populate with summaries and links
    \item Test search functionality
    \item Announce to team with quick training
    \item Collect feedback for first two weeks
\end{enumerate}

\section{Project 5: Meeting Intelligence System}

\textbf{Business Value:} Never lose meeting context; clear action items; faster follow-up

\textbf{Skills Needed:} Intermediate

\textbf{Tools Required:} Transcription tool (Otter.ai, Fireflies, etc.); ChatGPT or Claude

\textbf{Timeline:} 2-3 weeks

\subsection{Week 1: Set Up Transcription}

\textbf{Day 1-2: Choose and Configure Tool}
\begin{itemize}
    \item Select transcription tool (Otter.ai, Fireflies, Microsoft Teams transcription)
    \item Configure for your meeting platform
    \item Test with sample meeting
    \item Verify accuracy is acceptable
\end{itemize}

\textbf{Day 3-5: Create Processing Templates}

\begin{promptexample}[Meeting Summary Prompt]
Meeting transcript:
[paste transcript]

Create a meeting summary with:

1. \textbf{Overview} (2-3 sentences: what was this meeting about?)

2. \textbf{Decisions Made}
- [Decision 1]
- [Decision 2]
(Include who made the decision if clear from transcript)

3. \textbf{Action Items}
| Action | Owner | Due Date |
| --- | --- | --- |
| [Action 1] | [Name] | [Date if mentioned] |

4. \textbf{Key Discussion Points}
- [Point 1]
- [Point 2]

5. \textbf{Open Questions / Parking Lot}
- [Question 1]
- [Question 2]

6. \textbf{Next Steps}
- [Next step 1]
- [Next step 2]

Only include information actually discussed. Do not invent details.
\end{promptexample}

\subsection{Week 2: Process and Distribute}

\begin{enumerate}
    \item Process each meeting within 24 hours
    \item Send summary to attendees
    \item Track action items in project management system
    \item Follow up on overdue items
\end{enumerate}

\begin{tip}[Making It Stick]
The value of meeting intelligence comes from consistency. Assign one person to process every meeting. Make it a non-negotiable part of the meeting workflow.
\end{tip}

\subsection{Measurement}

\begin{table}[htbp]
\centering
\begin{tabular}{lll}
\toprule
\textbf{Metric} & \textbf{How to Measure} & \textbf{Target} \\
\midrule
Time to distribute notes & Track first 10 meetings & $<$4 hours \\
Action item completion & Track in PM tool & +20\% \\
Team satisfaction & Survey after 1 month & $>$4/5 \\
\bottomrule
\end{tabular}
\caption{Metrics for meeting intelligence project}
\end{table}

\section{Getting Started: Your First Week}

Do not try to implement all five projects at once. Here is how to start:

\textbf{Day 1-2:} Review all five projects. Which addresses your biggest pain point?

\textbf{Day 3:} Choose one project. Commit to it.

\textbf{Day 4-5:} Complete the first phase of your chosen project.

\textbf{Week 2:} Continue implementation. Document what works and what does not.

\textbf{Week 3-4:} Complete the project. Measure results.

\textbf{Week 5+:} Decide whether to scale this project or start another.

\begin{keyinsight}
One well-implemented AI project teaches you more than reading about ten. Start with the project that solves a real problem you face every week. The learning compounds.
\end{keyinsight}

\section{Summary}

These five projects---email templates, report automation, feedback analysis, knowledge base creation, and meeting intelligence---represent low-risk, high-value starting points for AI adoption. Each can be implemented by non-technical teams using existing AI tools.

The key to success is not the sophistication of the AI. It is the clarity of the problem, the discipline of the implementation, and the consistency of the measurement.

\begin{exercise}
Select one of the five projects and complete Week 1 within the next seven days. Document your experience: what worked, what was harder than expected, what you would do differently.
\end{exercise}

\begin{exercise}
After completing one project, calculate the actual ROI. How much time did it save? What was the implementation cost (your time)? Was it worth it?
\end{exercise}
