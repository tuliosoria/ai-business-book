\chapter{Presence: The Skill You Have to Practice}

\epigraph{The present moment is filled with joy and happiness. If you are attentive, you will see it.}{Thich Nhat Hanh}

\section{The Myth of Quality Time}

There's a comforting myth that busy fathers tell themselves: ``I may not have much time, but I make sure it's quality time.''

The problem with this myth is that children don't experience ``quality time'' as a category. They experience time. They don't know that Monday evening was supposed to be special because daddy carved out 30 minutes. They know that daddy was there---or wasn't.

More importantly, ``quality'' moments rarely arrive on schedule. The deepest connections happen in the middle of ordinary moments: during a diaper change, on a random Tuesday, in the quiet after a feeding. You can't manufacture these. You can only be present enough to notice them.

This chapter is about presence---the skill of actually being where you are, with the person you're with, fully engaged.

\section{The Distracted Father}

I'll confess my failure first.

In the early weeks, I was physically present but mentally elsewhere. I'd hold my son while checking my phone. I'd do tummy time while thinking about work. I'd rock him to sleep while composing emails in my head.

I was there, but I wasn't \textit{there}. My body was in the room; my attention was scattered across a dozen other concerns.

This is the default state for most of us. We're constantly pulled in multiple directions. The baby is just one demand among many.

But the baby knows. They know when you're distracted. They feel the difference between held by someone present and held by someone absent.

\begin{keyinsight}
Your physical presence means little without your attention. A distracted father is a half-present father. Children need you fully there, not just physically there.
\end{keyinsight}

\section{What Presence Actually Means}

Presence is:
\begin{itemize}
\item Attention without division---fully focused on what's in front of you
\item Awareness of the current moment, not rehearsing the past or future
\item Engagement with your senses---what you see, hear, feel right now
\item Emotional availability---open to whatever the moment brings
\item Non-judgment---accepting what is rather than wishing it were different
\end{itemize}

Presence is not:
\begin{itemize}
\item A passive state (it requires effort)
\item The same as relaxation (you can be present while active)
\item Perfection (you'll get distracted; you come back)
\item All-or-nothing (some presence is better than none)
\end{itemize}

\section{The Phone Problem}

Let's name the obvious obstacle: your phone.

The phone is a portal to infinite distraction. Every notification, every social media scroll, every news check pulls you out of the present moment and into an elsewhere. And unlike other distractions, the phone is always with you, always beckoning.

Research shows that the mere presence of a phone---even face-down and silent---reduces cognitive capacity and connection. Your brain knows it's there, knows it could buzz, maintains a background awareness that divides attention.

For meaningful presence with your baby, the phone needs boundaries.

\begin{practicaltip}[The Phone Boundary System]
Try these concrete boundaries:
\begin{itemize}
\item Phone stays in another room during designated baby time
\item Phone on airplane mode during feeds and bedtime routine
\item No phone during the first hour after waking or last hour before bed
\item One designated time per day for catching up on messages and news---not constant checking
\item Remove social media apps from your phone entirely
\end{itemize}
Start with one boundary. Once it's habit, add another.
\end{practicaltip}

\section{Practicing Presence}

Presence is a skill, like shooting free throws or playing piano. It improves with practice. Here are exercises:

\textbf{The Five Senses Check.}
When holding your baby, deliberately notice:
\begin{itemize}
\item What do you see? (Their face, their fingers, how the light falls)
\item What do you hear? (Their breathing, small noises, ambient sounds)
\item What do you feel? (Their weight, warmth, softness)
\item What do you smell? (Baby smell is real---notice it)
\item What textures are you touching?
\end{itemize}
This sensory inventory brings you back to the present.

\textbf{The Breath Anchor.}
When you notice your mind wandering, take three deep breaths. With each exhale, release the distraction. Return attention to whoever is in front of you.

\textbf{The ``Just This'' Practice.}
Whatever you're doing---changing a diaper, making a bottle, rocking to sleep---say to yourself: ``Just this.'' Not this plus planning tomorrow. Not this while solving a problem. Just this moment, this action, this person.

\textbf{The Eye Contact Exercise.}
During awake periods, spend time just looking at your baby while they look at you. No agenda. No entertainment. Just mutual gaze. This is how attachment forms.

\begin{realstory}[The Midnight Revelation]
It was 3 a.m. I was exhausted, resentful, wishing the baby would just sleep so I could sleep. I was mentally elsewhere---calculating how many hours until morning, dreading the next wakeup.

Then I looked down. My son was looking up at me. Not crying. Just looking. And for some reason, I stopped. I stopped the mental chatter, the resentment, the wishing for elsewhere.

I just looked back at him. We stayed like that for maybe two minutes.

In that moment, I understood something: these 3 a.m. sessions wouldn't last forever. Someday, probably soon, he wouldn't need me like this. The exhaustion would pass, but so would these quiet, dark, intimate moments.

I still struggled with night duty. But after that night, I tried harder to be there when I was there.
\end{realstory}

\section{The Stoic Practice of Attention}

The Stoics had a practice called \textit{prosoche}---attention. It meant maintaining awareness of your thoughts, judgments, and actions, living deliberately rather than reactively.

Marcus Aurelius wrote: ``Never value anything as profitable that compels you to break your promise, lose your self-respect, hate any man, suspect, curse, act the hypocrite, or desire anything that needs walls or curtains.''

Applied to fatherhood: Never value anything---work, phone, personal goals---so highly that you sacrifice the opportunity to be present with your child. These moments are not renewable. Your attention is the most valuable thing you can give.

\section{The Mundane Is the Sacred}

Here's the shift that changes everything: recognizing that the mundane moments \textit{are} the important moments.

We tend to think significance happens at special occasions---birthdays, milestones, holidays. The rest is filler, maintenance, getting through.

But to a baby, every moment is significant. There are no ordinary days. The feeding at 2 p.m. on a random Wednesday is as real as Christmas morning.

Your presence during the mundane moments is what your child actually experiences as love. Showing up consistently, day after day, for the unremarkable moments---this is what builds the bond.

\begin{keyinsight}[The Compound Interest of Presence]
Presence compounds. Each moment of genuine connection builds on the last. Over time, these accumulated moments become the foundation of your relationship. You can't deposit presence in one big lump; you have to make regular small deposits.
\end{keyinsight}

\section{When Presence Is Hard}

Some moments make presence easy: your baby's first laugh, peaceful snuggles, the joy of watching them discover something new.

But presence is also required during the hard moments: the endless crying, the diaper blowout, the 4 a.m. wakeup when you've already been up twice. Presence when you're exhausted, frustrated, or wishing you were anywhere else.

This is where presence becomes discipline rather than pleasure. You show up not because it feels good but because it's right.

\textbf{Strategies for hard-moment presence:}
\begin{itemize}
\item Acknowledge your feelings: ``I'm exhausted and frustrated. And I'm still here.''
\item Remember impermanence: ``This moment will pass. Right now, this is what's needed.''
\item Anchor to purpose: ``I'm doing this because I love my child.''
\item Accept the difficulty: Don't add resistance to suffering. The situation is hard; your judgment that it ``shouldn't be hard'' makes it worse.
\end{itemize}

\section{The Long Game of Presence}

Your baby won't remember these early months. No specific memories will form. So why does presence matter?

Because presence is not about creating memories. It's about creating attachment. The sense that ``I am safe, I am loved, I can trust'' forms in these early interactions. Your baby is learning, at a preverbal level, whether the world is responsive to their needs.

And you are forming habits. The patterns of presence or absence you establish now will continue. The distracted father of a newborn becomes the distracted father of a toddler, a child, a teenager. The present father of a newborn has practiced the skill that will matter for the next twenty years.

\section{The Christian Frame}

The Christian spiritual tradition has much to say about presence.

Jesus told Martha, busy with preparations, that Mary had ``chosen the better part'' by sitting at his feet, fully present (Luke 10:42). The contemplative traditions emphasize attention, awareness, and being fully present to God and others.

Brother Lawrence, a 17th-century monk, practiced what he called ``the presence of God''---maintaining awareness of God during even the most mundane tasks. Peeling potatoes was sacred work if done with attention and love.

Your diaper changes, your night feedings, your soothing of cries---these can be sacred work. Not despite their ordinariness but because of it. God is present in the present moment. When you are present, you encounter the divine in the everyday.

\begin{reflection}
When was the last time you were fully present---completely absorbed in the moment without division? What were the conditions? How can you create more of those conditions in your daily life with your baby?
\end{reflection}

\section{Building a Presence Practice}

Presence won't happen by accident. You need to build it deliberately.

\textbf{Daily practice:}
\begin{itemize}
\item Choose one daily activity to practice presence during (morning feed, bath time, bedtime routine)
\item During that activity, phone goes away, and attention stays on the baby
\item When you notice distraction, gently return attention without self-criticism
\item Over time, expand to additional activities
\end{itemize}

\textbf{Weekly practice:}
\begin{itemize}
\item One extended block (an hour or more) of undivided presence
\item Nothing else on the agenda---just being with your child
\item Notice what arises: restlessness, boredom, connection, joy
\end{itemize}

\textbf{Ongoing cultivation:}
\begin{itemize}
\item Regular meditation or prayer practice builds attention skills
\item Notice patterns: when is presence easiest? Hardest?
\item Extend presence practice beyond baby time to other relationships
\end{itemize}

\section{The Gift of Your Attention}

In an economy of distraction, attention is the scarcest resource. Everyone and everything competes for it. Your employer wants it. Your phone wants it. Your to-do list wants it.

When you give your baby your full attention, you're giving them something precious. You're saying, through action: ``You are worth my most valuable resource. You are worth being fully here for.''

This is the gift of presence. Not your money, not your accomplishments, not your plans for their future. Just you, here, now, paying attention.

It's the gift they need most. And it's within your power to give it.
